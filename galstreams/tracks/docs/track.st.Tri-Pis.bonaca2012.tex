The celestial track for the stream was implemented from the polynomial fit provided by \citet{Bonaca2012} for the Triangulum stream in their Eq. 1:
\begin{eqnarray*}
\delta = -4.4\alpha + 128\fdg5
\end{eqnarray*}

with $\alpha \in [+21^\circ,+24^\circ]$, as explicitly reported by the authors. \citet{Bonaca2012} report a mean heliocentric 
distance of 26~kpc for the stream and a width of $0\fdg2$. 

This coincides with the feature named 'stream a' in \citet{Grillmair2012}.\footnote{Since this reference is a conference proceedings and not a full length paper, we have chosen not to cite it as a discovery reference.}
Soon after the discovery by \citet{Bonaca2012}, \citet{Martin2013} reported the independent discovery of the same structure, based on radial velocity data, naming it the Pisces stream. This detection spans the easternmost $\sim1^\circ$ of the $\sim13^\circ$ track detected by \citet{Bonaca2012}. Based on their spectroscopic metallicity measurement of $\FeH=-2.2$, \citet{Martin2013} find a distance of 35~kpc to the stream, much larger than the 26~kpc found by Bonaca et al's, based on a significantly larger metallicity of $\FeH\sim -1$ estimated from isochrone fitting. Here we will adopt the larger distance estimate of 35~kpc for the full track, as it is based in the more reliable spectroscopic measurement of the metallicity.

The radial velocity track was implemented with the mean of the radial velocities from \citet{Martin2013} because the available data is too noisy and its along-stream span too short to justify higher order fitting. We set this mean value as the radial velocity for track in the range $23\fdg2<\alpha<24\fdg2$ spanned by the observations; outside this range we have set the radial velocity track to zero. The radial velocities reported by \citet{Martin2013} are in the GSR. To revert back to the heliocentric frame and compute the observed radial velocity we have assumed  a solar peculiar velocity with respect to the LSR $(U,V,W)_\odot=(11.1,12.24,7.25)$~km/s \citet{Schoenrich2010} and  $V_{LSR}=220$~km/s \citep{DehnenBinney1998}, since the solar parameters used to convert to the GSR were not reported by the authors.

Since the previous version of galstreams this stream has been referred to as Triangulum-Pisces (in short Tri-Pis), following \citet{GrillmairCarlin2016}. We have kept this naming convention to account for the two independent discoveries.
