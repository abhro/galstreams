The celestial track for Yangtze was implemented from the polynomial fit provided by \citet{Yang2023b} in their Eq.~3:
\begin{eqnarray*}
\alpha = (5.194\times10^{-3})\delta^2 + (4.685\times10^{-2})\delta + 146.141
\end{eqnarray*}

with $\delta \in [-13^\circ,14^\circ]$. The authors report a mean distance modulus of 14.8, corresponding to a heliocentric distances of 9.2~kpc, used to implement the distance track. Note, however, that for such a long stream this is probably just an approximation. The proper motion track is implemented assuming the mean proper motions of $\mu_\alpha^*,\mu_\delta=(-0.7,1.1)$~mas/yr reported by the authors for the full track (the corresponding InfoFlag bit is set to 2 as a warning). Figure~3 in \citet{Yang2023b} shows there may be a small proper motion gradient along the stream. The Yangtze stream is tentatively associated by the authors to the Pal~1 cluster and the AntiCentre Stream (ACS). As cited in \citet{Mateu2023) for ACS there is fair consensus that ACS is not a tidal stream, but is comprised by population perturbed out of the Galactic disc's plane \citep[for more details see][and references therein]{Laporte2019a,Laporte2019b,Ramos2021}.
